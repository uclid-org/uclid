This example corresponds to Figure 11 in our submitted paper. It describes an idealized enclave platform with a set of primitive operations parameterized by three different adversaries.

TAP is modeled as a transition system that executes the trusted enclave or an adversary operation at each step atomically. The enclave is allowed to execute a subset of the primitive operations such as \texttt{pause} and \texttt{exit}, and the adversary is allowed to execute instructions such as \texttt{launch} with any arbitrary input arguments. Prove a hyperproperty like integrity requires reasoning about two TAP traces, which is demonstrated in UCLID5 as instantations of the tap model as shown in the directory \texttt{proofs/integrity-proof.ucl}. Running UCLID5 should verify all the invariants in the integrity proof. 
