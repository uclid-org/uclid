\chapter{Basics: Types and Statements}
\begin{uclidlisting}[htbp]
    \lstinputlisting[language=uclid,style=uclidstyle]{../examples/tutorial/ex2.1-alu.ucl}
    \caption{Model of a simple ALU}
\label{ex:alu}
\end{uclidlisting}

This chapter will provide an overview of \uclid{}'s type system and modelling features. Let us start with Example~\ref{ex:alu}, a model of a simple arithmetic logic unit (ALU).

\section{Types in \uclid{}}

Types supported by \uclid{} are of the following kinds:
\begin{enumerate}
    \item \keyword{integer}: the type of mathematical integers.
    \item \keyword{boolean}: the Boolean type. This type has two values: \keyword{true} and \keyword{false}.
    \item \keyword{bv\textit{W}}: The family of bit-vector types parameterized by their width ($W$).
    \item \keyword{enum}: enumerated types.
    \item Float types: Support fp(a)_(b), which is defined as float numbers with an $a$ bit exponent and $b$ bit signficiant and $1$ sign bit. Also, \keyword{half},\keyword{single},\keyword{double} are supported.
    \item Tuples and records.
    \item Array types.
    \item Uninterpreted types.
    \item Groups: A finite group of variables of a single type. Groups are not modifiable once declared, i.e., they are a static size. 
\end{enumerate}

An enumerated type is used in line~2 of Example~\ref{ex:alu}. This declares a \textit{type synonym}: \codelike{cmd_t} is an alias for the enumerated type consisting of three values: \codelike{add}, \codelike{sub} and \codelike{mov_imm}. The input \codelike{cmd} is then declared to be of type \codelike{cmd_t} on line~6.

The input \codelike{valid} is of type \keyword{boolean}. Register indices \codelike{r1} and \codelike{r2} are bit-vectors of width 3 (\keyword{bv3}), while \codelike{immed}, \codelike{r1val} and \codelike{r2val} are bit-vectors of width 8 (\keyword{bv8}).

Line~3 declares a type synonym for a \keyword{record}. It declares \codelike{result_t} as consisting of two fields: a Boolean field \codelike{valid} and a bit-vector field \codelike{value}. The output \codelike{result} is declared to be of type \codelike{result_t} on Line~9.

The final point of interest in this example, type-wise, is line~10. The state variable \codelike{regs} is declared to be of type array: indices to the array are of type \codelike{bv3} and elements of the array are of type \codelike{bv8}. This is used to model an 8-entry register file, where each register is a bit-vector of width 8.

Uninterpreted functions can be declared in \uclid{} using the
\codelike{function} declaration. These functions are typed,
mapping a tuple of typed arguments to a return type.
For example, the following uninterpreted function models
the mapping of an instruction to an opcode in a simple
CPU model given later in this tutorial as Example~\ref{ex:cpu-cpu}.
\begin{lstlisting}[language=uclid,style=uclidstyle]
  function inst2op   (i : word_t) : op_t;
\end{lstlisting}

Two further types of functions without definitions can be declared in \uclid{} using the
\codelike{function} declaration. 
First a synthesis function, declared using the synthesis keyword. Given a SyGuS solver using the command line option \codelike{-y "solver path"},
\uclid{} will construct
a SyGuS-IF file that synthesises bodies for any synthesis function in the \uclid{} file
\begin{lstlisting}[language=uclid,style=uclidstyle]
  synthesis function inst2op   (i : word_t) : op_t;
\end{lstlisting}

The second function extension is an oracle function, declared using the oracle keyword. An oracle function is declared along with the name of an external executable binary which implements the function. Given an Satisfiability Modulo Theories and Oracles (SMTO) solver using the command line option \codelike{-s "solver path"}, or a Synthesis Modulo Oracles (SyMO) solver using the command line option \codelike{-y "solver path"},
\uclid{} will construct
an SMTO/SyMO file for the verification/synthesis query including the oracle function. For more details on Synthesis and Satisfiability Modulo Oracles see \url{https://arxiv.org/abs/2107.13477}.
\begin{lstlisting}[language=uclid,style=uclidstyle]
  oracle function inst2op  [binaryName] (i : word_t) : op_t;
\end{lstlisting}

 
{\em Symbolic constants} can be declared using a 
\codelike{const} declaration, as follows:
\begin{lstlisting}[language=uclid,style=uclidstyle]
  const w0 : word_t;
\end{lstlisting}


Finally, a useful feature of UCLID is groups. Groups are declared using the keyword \codelike{group}. All members of a group must have the same type, but they can be literals or variables that are already declared and in scope at the point of the group declaration.
\begin{lstlisting}[language=uclid,style=uclidstyle]
  group MyGroup :  integer = {0, 1, 2, 3};
  group AnotherGroup: integer = {a, b, c, 7};
\end{lstlisting}


\section{Statements in \uclid{}}

Computation in \uclid{} can be either procedural (sequential) or parallel (concurrent). Procedural computation is performed by defining a \keyword{procedure} (and in the \codelike{init} block) while parallel computation occurs in the \keyword{next} block.

\subsection{Parallel vs. Procedural Assignments}
Assignments inside procedures and the \codelike{init} block are called \textbf{procedural assignments} and must be of the form \codelike{variable = expression;} Assignments inside \codelike{next} block are \textbf{parallel assignments} and must be of the form \codelike{variable' = expression;}. Mathematically, parallel assignments compute the next state of the transition system described by the model.

An example showing the use of sequential assignments is the following:

%\begin{uclidlisting}[htbp]
\begin{lstlisting}[language=uclid,style=uclidstyle]
  x = 1;
  x = x + 2;
  x = x + 3;
\end{lstlisting}

	In this example, \codelike{x} is assigned sequentially.  Recall that these procedural assignments \emph{must} appear inside a procedure or in the \codelike{init} block. Executing these three statements will result \codelike{x} having the value 6.

    In contrast, the following sequence of parallel assignments is \textbf{not} allowed and will result in a compiler error.
\begin{lstlisting}[language=uclid,style=uclidstyle]
  // Error, will not compile.
  x' = 1;
  x' = x + 2;
  x' = x + 3;
\end{lstlisting}
Only a single parallel assignment to a state/output variable is allowed in a code block. Furthermore, since parallel assignments are computed in data-flow order, the order in which they are specified does not matter. This means that the following two snippets of code are equivalent:

\begin{lstlisting}[language=uclid,style=uclidstyle]
next {
  x' = x + 1;
  y' = x' + 1;
}
\end{lstlisting}

\begin{lstlisting}[language=uclid,style=uclidstyle]
next {
  y' = x' + 1;
  x' = x + 1;
}
\end{lstlisting}

    \uclid{} will determine that since \codelike{y'} depends on the value of \codelike{x'}, \codelike{x'} has to be computed first. This value is then used in the computation of \codelike{y'}. This is regardless of the order in which these assignments appear in the \codelike{next} block.

    Note also that the assignment to \codelike{x'} uses the value of the variable \codelike{x} \emph{at the beginning of the current step} of the transition system (i.e., the ``old'' value of \codelike{x}). In contrast the assignment to \codelike{y'} uses the ``new'' value of \codelike{x}, which is the value of x at the \emph{end} of this step of the transition system. It is important to think carefully about which version of a variable (\codelike{var} or \codelike{var'}) must be used in a particular assignment.
%\caption{Example of sequential assignment.}
%\label{ex:seq-assignment}
%\end{uclidlisting}

    %\lstinputlisting
\subsection{Procedures}
Example~\ref{ex:alu} demonstrates how sequential computation is used in concert with parallel computation. %The \codelike{procedure} \codelike{set\_init\_state} (lines~13--18) is used to initialize the values of the registers (state variable \codelike{regs}), the variables \codelike{cnt} and \codelike{result}. Since this procedure updates the module's state variables, a \codelike{modifies} clause (line~14) is required to explicitly specify that these updates are intended. The procedure is called on line~21 in the \codelike{init} block. Updates to a state variable not mentioned in a modifies clause will result in a compilation error.
In Example~\ref{ex:alu}, consider \codelike{procedure} \codelike{exec\_cmd} which executes a single ALU command and returns (line 23) a single value of type \codelike{result\_t}. The procedure is invoked on line~39 in the \codelike{next} block, and its return value is assigned to the output variable \codelike{result}. Note we are again using the notation \codelike{result'} to refer to parallel assignment. Since this procedure updates the \codelike{reg} state variable, a \keyword{modifies} clause must be used to declare this fact (line 24).

We emphasize that assignments inside procedures do not assign primed
variables. However, if a state variable is defined to be modified by
a procedure (mentioned in its modifies clause), then its next-state
value is the value that variable has upon return from that procedure.
Put another way, the post-state of the procedure determines the
next-state assignment of all state variables modified by it.
In our example, \codelike{procedure} \codelike{exec\_cmd} modifies
the state variable \codelike{regs}, and thus determines its next-state
value.

\subsection{Macro Definitions}

\uclid{} also supports the definition of macro expressions using
the \keyword{define} statement. 
Line~20 of Example~\ref{ex:alu} illustrates this construct.
Macro definitions are useful, as in C, to define expressions
over arguments that are instantiated in multiple places, or
which help make the code more readable.


\subsection{For Loops}
The procedure \codelike{set\_init\_state} uses a \keyword{for} loop to initialize each value in the array \codelike{regs} to the bit-vector value 1.\footnote{\codelike{1bv8} here refers to the bit-vector value 1 of width 8.} The loop iterates over the values between 0 and 7 (both-inclusive).

The range over which a \keyword{for} loop iterates must be defined by two numeric literals.

Alternatively, the loop iterates may be defined by two macro definitions whose expressions are numeric literals. However, this requires the for statement to be declared with a typed iterator.

For example, if the following macro definitions are at the module level,
\begin{lstlisting}[language=uclid,style=uclidstyle]
  define begin() : bv3 = 0bv3;
  define end()   : bv3 = 7bv3;
\end{lstlisting}
we may alternatively write the for loop in \codelike{set\_init\_state} in the following way:
\begin{lstlisting}[language=uclid,style=uclidstyle]
  for (i : bv3) in (begin(), end()) { regs[i] = 1bv8; }
\end{lstlisting}

\subsection{If and Case Statements}
Also worth pointing out are the \keyword{if} statement that appears on line~27, and the \keyword{case} statement that appears on line~29. Syntax for \keyword{if} statements should be familiar.

\keyword{case} statements are delimited by \keyword{case} and \keyword{esac} and contain within them a list of boolean expressions and associated statement blocks. These expressions are evaluated in the order in which appear, and if any of them evaluate to \codelike{true}, the corresponding block is executed. If none of the case-expressions evaluate to \codelike{true}, nothing is executed. The keyword \keyword{default} can be used as a ``catch-all'' case like in C/C++.

\subsection{Expressions}

The syntax for expressions in \uclid{} is similar to languages like C/C++/Java. Index \codelike{i} of array \codelike{regs} is accessed using the syntax \codelike{regs[i]}. Field \codelike{value} in the record \codelike{result} is accessed as \codelike{result.value}.

\subsection{Quantifiers}

\uclid{} supports universal and existential quantifiers, which can be used in any place where a predicate would be permitted, for instance, in assumptions:
\begin{lstlisting}[language=uclid,style=uclidstyle]
  assume (forall (ri : robind_t) :: ops[ri] == no_op);
  assume (exists (i : integer) :: (i > 0) && (i < 11) && numbers[i] != some_int);
\end{lstlisting}
\uclid{} Also supports finite universal quantification, where we universally quantify over a finite group type. These quantifiers are expanded to a conjunction before being passed to the underlying SMT solver.
\begin{lstlisting}[language=uclid,style=uclidstyle]
  group myGroup : integer = {0, 1, 2, 3};
  property test : finite_forall (int : integer) in myGroup :: int < 4;
\end{lstlisting}


%-----------------------------------------------------
\section{An Illustrative Example}

This section briefly describes the execution semantics of Example~\ref{ex:alu}.

\subsection{Initialization}
Execution of the model in Example~\ref{ex:alu} starts with the \keyword{init} block. This block invokes \codelike{set\_init\_state} and assigns initial values to \codelike{regs}, \codelike{cnt} and \codelike{result.value}. The other variables (e.g. \codelike{r1val} and \codelike{r2val}) are not assigned to in the \keyword{init} block and will be initialized non-deterministically.

\subsection{Next State Computation}
The next state of each state variable in the model is computed according to the \keyword{next} block. Any variables not assigned to in the \keyword{next} block retain their ``old'' values.

The \keyword{input} variables of the model are assigned (possibly different) non-deterministic values for each step of the transition system. These values can be controlled by using assumptions. Indeed, the model uses the two assumptions on lines 43--44 to constrain the input to the ALU to always be an \codelike{add} operation, where both operands refer to register index 0.

\subsection{Verification}
As in Example~\ref{ex:fib-model}, the verification script in Example~\ref{ex:alu} unrolls the transition system for 5 steps and checks if the \keyword{invariant} on line~45 is violated in any of these steps.

\subsection{Running \uclid{}}

Running \uclid{} on Example~\ref{ex:alu} produces the following output.

\begin{Verbatim}[frame=single, samepage=true]
$ uclid examples/tutorial/ex2.1-alu.ucl
Successfully parsed 1 and instantiated 1 module(s).
6 assertions passed.
0 assertions failed.
0 assertions indeterminate.
Finished execution for module: main.
\end{Verbatim}

\uclid{} is able to prove that the \keyword{invariant} on line~45 holds for all states reachable within 5 steps of the initial state, under the assumptions specified in lines~43--44.
